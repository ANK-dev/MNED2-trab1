\chapter{Desenvolvimento}
Neste capítulo serão abordados os passos e métodos utilizados para se obter a
solução numérica do problema proposto.

\section{Condições Inicial e de Contorno}
A resolução de qualquer equação diferencial parcial (EDP) requer a determinação
de sua condição(ões) inicial(ais) e de contorno. No caso da EDP discretizada
(Eq.\ \ref{eq. final}), o mesmo se aplica.
\[
    Q_i^{n+1} = Q_i^n - \Delta t
    \left[
    \bar{u}\frac{(Q_i^n - Q_{i-1}^n)}{\Delta x}
    -
    \alpha\frac{(Q_{i+1}^n - 2Q_i^n + Q_{i-1}^n)}{\Delta x^2}
    \right]
\]

Nota-se que há uma dependência temporal do termo futuro $Q_i^{n+1}$ em relação
aos termos $Q$ presentes em $n$, e seus vizinhos espaciais $i$, ${i\pm1}$. A
partir desta relação é possível perceber que, para se calcular o primeiro
termo, $Q^1$, é necessário um termo de partida, $Q^0$. Se tem, assim, a
necessidade do estabelecimento de uma condição inicial. Neste trabalho,
considera-se uma concentração inicial, $c_{\text{ini}}$, constante para toda a
malha.

Mudando o foco para os volumes da malha, nos volumes $i=0$ e $i=6$ há uma
dependência de termos localizados além do seu domínio --- $Q_0^n$ e
$Q_{nx+1}^n$, respectivamente. Para resolver este problema, neste trabalho
foram adotadas as seguintes condições de contorno:

% Equações lado-a-lado
\noindent
\begin{minipage}{.5\linewidth}
    \begin{equation}
        c(x=0,t) = c_{\text{inj}}
    \end{equation}
\end{minipage}%
\begin{minipage}{.5\linewidth}
    \begin{equation}
        \left(\frac{\partial c}{\partial x}\right)_{x=L_x}^t = 0
    \end{equation}
\end{minipage}

\bigskip

Aliado a estas condições, utiliza-se o conceito de \emph{volumes fantasmas},
para a definição destas condições no discreto. É possível redefinir as
condições de contorno como uma média entre os dois volumes adjacentes. Ao se
realizar tal construção para a fronteira esquerda, obtém-se,
\begin{center}
    \textit{--- Inserir gráfico sobre células fantasmas ---}
\end{center}
\[
    c_{\text{inj}} = \frac{Q_0^n + Q_1^n}{2}
\]
\begin{equation}\label{cont. esq}
    Q_0^n = 2c_{\text{inj}} - Q_1^n
\end{equation}
Analogamente, para a fronteira direita, obtém-se,
\[
  \left(\frac{\partial c}{\partial x}\right)_{x=L_x}^t
  \approx
  \frac{Q_{nx+1}^n - Q_{nx}^n}{\Delta x} = 0
\]
\begin{equation}\label{cont. dir}
    Q_{nx+1}^n = Q_{nx}^n
\end{equation}

A partir de ambas as relações, definem-se as equações discretas. Para o contorno
esquerdo,
\[
    Q_1^{n+1} = Q_1^n - \Delta t
    \left[
    \bar{u}\frac{(Q_1^n - (2c_{\text{inj}} - Q_1^n))}{\Delta x}
    -
    \alpha\frac{(Q_2^n - 2Q_1^n + (2c_{\text{inj}} - Q_1^n))}{\Delta x^2}
    \right]
\]
\begin{equation}
    Q_1^{n+1} = Q_1^n - \Delta t
    \left[
    \bar{u}\frac{(2Q_1^n - 2c_{\text{inj}})}{\Delta x}
    -
    \alpha\frac{(Q_2^n - 3Q_1^n + 2c_{\text{inj}})}{\Delta x^2}
    \right]
\end{equation}
e para o contorno direito,
\[
    Q_{nx}^{n+1} = Q_{nx}^n - \Delta t
    \left[
    \bar{u}\frac{(Q_{nx}^n - Q_{nx-1}^n)}{\Delta x}
    -
    \alpha\frac{(Q_{nx}^n - 2Q_{nx}^n + Q_{nx-1}^n)}{\Delta x^2}
    \right]
\]
\begin{equation}
    Q_{nx}^{n+1} = Q_{nx}^n - \Delta t
    \left[
    \bar{u}\frac{(Q_{nx}^n - Q_{nx-1}^n)}{\Delta x}
    -
    \alpha\frac{(Q_{nx-1}^n - Q_{nx}^n)}{\Delta x^2}
    \right]
\end{equation}

\section{Consistência, Convergência e Estabilidade}
A análise da consistência, convergência e estabilidade de uma EDP tem como
finalidade garantir que a solução numérica do problema --- calculada por
algoritmos --- se aproxime o máximo possível da solução real, com algumas
observações.

\subsection{Consistência}
Se trata da equivalência da forma algorítmica da EDP em relação a sua forma
analítica. Um método numérico é dito \emph{consistente} quando, através
de operações algébricas, é possível recuperar a EDP original.

\subsection{Convergência}
Se trata da aproximação dos valores numéricos do algoritmo à solução analítica
da EDP, dado um certo número de iterações. Um método numérico é dito
\emph{convergente} quando este sempre irá tender aos valores da solução, não
importando o número de iterações.

A análise direta da convergência de algoritmo é muito difícil, mesmo para os
casos mais fáceis. Uma possível saída para esse problema é utilizar o Teorema
da Equivalência de Lax, que diz:
\begin{displayquote}[Peter Lax]
    ``Para um problema linear de valor inicial bem-posto e um método de
    discretização consistente, estabilidade é condição necessária e suficiente
    para a convergência.''
\end{displayquote}
Como a Eq.\ \ref{eq. final} foi obtida a partir da EDP analítica, é certa a
sua consistência, restando assim, a determinação de sua estabilidade para a
garantia da convergência.

\subsection{Estabilidade}
Se trata do comportamento do algoritmo e seus valores numéricos frente aos
parâmetros de entrada. Um algoritmo \emph{estável} se comporta de maneira
esperada frente a uma faixa específica de valores de entrada.

Para se determinar a estabilidade da Eq.\ \ref{eq. final}, é possível
reescrevê-la para a análise através do método de Von Neumann.
\begin{equation}
    (Q^*)_i^{n+1} = (s + C)(Q^*)_{i-1}^n + (1 - 2s + C)(Q^*)_i^n +
    s(Q^*)_{i+1}^n
\end{equation}
onde $Q^*$ é solução numérica da EDP, sujeita aos erros de arredondamento do
computador, $s = \frac{\alpha\Delta t}{\Delta x^2}$ e $C = \frac{\bar{u}\Delta
t}{\Delta x}$ (número de Courant). Além disso, define-se uma nova equação,
\begin{equation}\label{eq. final erro}
    \xi_i^{n+1} = (s + C)\xi_{i-1}^n + (1 - 2s + C)\xi_i^n + s\xi_{i+1}^n
\end{equation}
onde $\xi_i^n = Q_i^n - (Q^*)_i^n$ é o erro numérico introduzido em cada ponto
da malha.

Seguindo o método de Von Neumann, é possível expandir cada termo $\xi_i^n$ em
uma série de Fourier,
\begin{equation}
    \xi_i^n = (G)^n e^{j \theta i}
\end{equation}
onde $j$ é a unidade imaginária e G é um fator de amplificação, de forma que,
\begin{equation}
    \xi_n^{n+1} = G\xi_i^n
\end{equation}
e a estabilidade é garantida se $|G| \leq 1$. Substituindo esses termos em Eq.\
\ref{eq. final erro}, obtém-se a seguinte equação:
\begin{equation}
    G^{n+1}e^{j \theta i} =
      (s + C)G^n e^{j \theta (i-1)}
      + (1 - 2s + C)G^n e^{j \theta i}
      + s G^n e^{j \theta (i+1)}
\end{equation}
Dividindo-a por $G^n e^{j \theta i}$,
\[
    G = (s + C)e^{-j \theta} + (1 - 2s + C) + se^{j \theta}
\]

A partir das relações de Euler, $e^{-j \theta} = \cos\theta - j\sin\theta$ \ \&
\ $e^j\theta = \cos\theta + j\sin\theta$, é possível reescrevê-la como,
\[
    G = (s + C)e^{-j \theta} + (1 - 2s + C) + se^{j \theta}
\]
\[
    G = 1 - (2s + C)(1 - \cos\theta) - jC\sin\theta
\]
tendo assim,
\begin{equation}
    |G| = \sqrt{[1 - (2s + C)(1-\cos\theta)]^2 + C^2\sin^2\theta}\ \leq\ 1
\end{equation}

O caso limite para esta construção ocorre quando $\sin\theta = \pm 1$ e
$\cos\theta = 0$. É necessário analisar quais são os valores de $s$ e $C$ que
mantém a veracidade da inequação.
\[
    |G| = \sqrt{[1 - (2s + C)(1)]^2 + C^2} \ \leq\ 1
\]
Se $2s + C \leq 1$ a condição é satisfeita. O que acontece, porém, no novo caso
limite de $2s + C = 1$?
\[
    |G| = \sqrt{(1-1)^2 + C^2} = \sqrt{C^2} = C
\]
Como $2s + C = 1$, C \textbf{tem} que ser menor que 1 para um $s \neq 0$. Desta
forma,
\[
    2\frac{\alpha\Delta t}{\Delta x^2} + \frac{\bar{u}\Delta t}{\Delta x} \leq 1
\]
\[
    \Delta t
    \left(
        \frac{2\alpha}{\Delta x^2} + \frac{\bar{u}\Delta t}{\Delta x}
    \right) \leq 1
\]
\begin{equation}
    \Delta t
    \leq
    \frac{1}{\frac{2\alpha}{\Delta x^2} + \frac{\bar{u}}{\Delta x}}
\end{equation}

Tem-se, então, um método \emph{condicionalmente estável}, ou seja, que depende
de uma faixa de valores para garantir a estabilidade de seu funcionamento.

\section{Programação}
Aliado destes conceitos, foi possível construir um programa em linguagem C que
calcula as concentrações para cada célula ao longo do tempo, exportando um
arquivo de texto com os resultados; este arquivo, então, é lido por um
\textit{script} Python que gera um gráfico correspondente.

O programa principal possui a seguinte estrutura, descrita em C:

\begin{Verbatim}[fontsize=\footnotesize]
// Vetores para concentração no tempo `n' e no tempo `n+1', respectivamente
double Q_old[];
double Q_new[];

// Inicializa ambos os vetores com os valores da concentração inicial (c_ini)
inicializaVetor(Q_antigo)
inicializaVetor(Q_novo)

// Cálculo de Q, iterado ao longo do tempo
do {

// Cálculo do volume da fronteira esquerda
Q_new[0] = Q_old[0] - Delta_t/Delta_x
           * (
               u_bar * (2*Q_old[0] - 2*c_inj)
              -
               alpha * (Q_old[1] - 3*Q_old[0] + 2*c_inj) / Delta_x
             );

// Cálculo de volumes do centro da malha
for (x = 1; x < nx - 1; ++x) {
    Q_new[x] = Q_old[x] - Delta_t/Delta_x
               * (
                   u_bar * (Q_old[x] - Q_old[x-1])
                  -
                   alpha * (Q_old[x+1] - 2*Q_old[x] - Q_old[x-1]) / Delta_x
                 );
}

// Cálculo do volume da fronteira direita
Q_new[x] = Q_old[x] - Delta_t/Delta_x
           * (
               u_bar * (Q_old[x] - Q_old[x-1])
              -
               alpha * (Q_old[x-1] - Q_old[x]) / Delta_x
             );

// Atualiza vetores antigos para a próxima iteração
for (x = 0; x < nx; ++x) {
    Q_old[x] = Q_new[x];
}

// Incrementa passo de tempo
} while ( (t += Delta_t) < t_final);

\end{Verbatim}

São definidos dois vetores, \verb|Q_old[]| e \verb|Q_new[]|, que correspondem
as concentrações $Q$ no tempo $n$ e $n+1$, respectivamente. Antes do cálculo
das concentrações, os vetores são inicializados, em um simples laço \verb|for|,
com os valores de concentração inicial $c_\text{ini}$.

A cada iteração do laço \verb|do-while|, o tempo \verb|t| é incrementado por
uma quantidade \verb|Delta_t|, que obedece as regras de estabilidade descritas
na seção anterior. Ao longo da iteração, o vetor \verb|Q_new[]| é calculado
para as fronteiras e para o centro da malha, em função de \verb|Q_old[]|. Antes
do fim da iteração, os vetores \verb|Q_old[]| são atualizados com os valores de
\verb|Q_new[]|, o tempo é incrementado, e então a nova iteração é iniciada.

Ao fim da execução, o vetor \verb|Q_new[]|, terá os resultados da concentração
de cada célula da malha, correspondente a cada índice do vetor, no tempo
\verb|t = t_final|. Os pares índice-concentração são exportados em um arquivo
de texto, linha-a-linha, para serem lidos e plotados pelo \textit{script}
Python.
