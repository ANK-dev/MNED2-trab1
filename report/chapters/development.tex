\chapter{Desenvolvimento}
Neste capítulo serão abordados os passos e métodos utilizados para se obter a
solução numérica do problema proposto.

\section{Condições Inicial e de Contorno}
A resolução de qualquer equação diferencial parcial (EDP) requer a determinação
de sua condição(ões) inicial(ais) e de contorno. No caso da EDP discretizada
(Eq.\ \ref{eq. final}), o mesmo se aplica.

\[
    Q_i^{n+1} = Q_i^n - \Delta t
    \left[
    \bar{u}\frac{(Q_i^n - Q_{i-1}^n)}{\Delta x}
    -
    \alpha\frac{(Q_{i+1}^n - 2Q_i^n + Q_{i-1}^n)}{\Delta x^2}
    \right]
\]

Nota-se que há uma dependência temporal do termo futuro $Q_i^{n+1}$ em relação
aos termos $Q$ presentes em $n$, e seus vizinhos espaciais $i$, ${i\pm1}$. A
partir desta relação é possível perceber que, para se calcular o primeiro
termo, $Q^1$, é necessário um termo de partida, $Q^0$. Se tem, assim, a
necessidade do estabelecimento de uma condição inicial. Neste trabalho,
considera-se uma concentração inicial, $c_{\text{ini}}$, constante para toda a
malha.

Mudando o foco para os volumes da malha, nos volumes $i=0$ e $i=6$ há uma
dependência de termos localizados além do seu domínio --- $Q_0^n$ e
$Q_{nx+1}^n$, respectivamente. Para resolver este problema, neste trabalho
foram adotadas as seguintes condições de contorno:

% Equações lado-a-lado
\noindent
\begin{minipage}{.5\linewidth}
    \begin{equation}
        c(x=0,t) = c_{\text{inj}}
    \end{equation}
\end{minipage}%
\begin{minipage}{.5\linewidth}
    \begin{equation}
        \left(\frac{\partial c}{\partial x}\right)_{x=L_x}^t = 0
    \end{equation}
\end{minipage}

\bigskip

Aliado a estas condições, utiliza-se o conceito de \emph{volumes fantasmas},
para a definição destas condições no discreto. É possível redefinir as
condições de contorno como uma média entre os dois volumes adjacentes. Ao se
realizar tal construção para a fronteira esquerda, obtém-se,

\begin{center}
    \textit{--- Inserir gráfico sobre células fantasmas ---}
\end{center}

\[
    c_{\text{inj}} = \frac{Q_0^n + Q_1^n}{2}
\]

\begin{equation}\label{cont. esq}
    Q_0^n = 2c_{\text{inj}} - Q_1^n
\end{equation}

Analogamente, para a fronteira direita, obtém-se,

\[
  \left(\frac{\partial c}{\partial x}\right)_{x=L_x}^t
  \approx
  \frac{Q_{nx+1}^n - Q_{nx}^n}{\Delta x} = 0
\]

\begin{equation}\label{cont. dir}
    Q_{nx+1}^n = Q_{nx}^n
\end{equation}

A partir de ambas as relações, definem-se as equações discretas. Para o contorno
esquerdo,

\[
    Q_1^{n+1} = Q_1^n - \Delta t
    \left[
    \bar{u}\frac{(Q_1^n - (2c_{\text{inj}} - Q_1^n))}{\Delta x}
    -
    \alpha\frac{(Q_2^n - 2Q_1^n + (2c_{\text{inj}} - Q_1^n))}{\Delta x^2}
    \right]
\]

\begin{equation}
    Q_1^{n+1} = Q_1^n - \Delta t
    \left[
    \bar{u}\frac{(2Q_1^n - 2c_{\text{inj}})}{\Delta x}
    -
    \alpha\frac{(Q_2^n - 3Q_1^n + 2c_{\text{inj}})}{\Delta x^2}
    \right]
\end{equation}

\noindent e para o contorno direito,

\[
    Q_{nx}^{n+1} = Q_{nx}^n - \Delta t
    \left[
    \bar{u}\frac{(Q_{nx}^n - Q_{nx-1}^n)}{\Delta x}
    -
    \alpha\frac{(Q_{nx}^n - 2Q_{nx}^n + Q_{nx-1}^n)}{\Delta x^2}
    \right]
\]

\begin{equation}
    Q_{nx}^{n+1} = Q_{nx}^n - \Delta t
    \left[
    \bar{u}\frac{(Q_{nx}^n - Q_{nx-1}^n)}{\Delta x}
    -
    \alpha\frac{(Q_{nx-1}^n - Q_{nx}^n)}{\Delta x^2}
    \right]
\end{equation}

\section{Consistência, Convergência e Estabilidade}
A análise da consistência, convergência e estabilidade de uma EDP tem como
finalidade garantir que a solução numérica do problema --- calculada por
algoritmos --- se aproxime o máximo possível da solução real, com algumas
observações.

\subsection{Consistência}
Se trata da equivalência da forma algorítmica da EDP em relação a sua forma
analítica. Um método numérico é dito \emph{consistente} quando, através
de operações algébricas, é possível recuperar a EDP original.

\subsection{Convergência}
Se trata da aproximação dos valores numéricos do algoritmo à solução analítica
da EDP, dado um certo número de iterações. Um método numérico é dito
\emph{convergente} quando este sempre irá tender aos valores da solução, não
importando o número de iterações.

\subsection{Estabilidade e Teorema de Lax}
Se trata do comportamento do algoritmo e seus valores numéricos frente aos
parâmetros de entrada. Um algoritmo \emph{estável} se comporta de maneira
esperada frente a uma faixa específica de valores de entrada.

A análise direta da estabilidade de algoritmo é muito difícil, mesmo para os
casos mais fáceis. Uma possível saída para esse problema é utilizar o Teorema
da Equivalência de Lax, que diz:

\begin{displayquote}[Peter Lax]
    ``Para um problema linear de valor inicial bem-posto e um método de
    discretização consistente, estabilidade é condição necessária e suficiente
    para a convergência.''
\end{displayquote}

