\chapter{Código Computacional}

\subsection{Código Principal}

\subsubsection{Arquivo \texttt{main.h}}

\begin{Verbatim}[fontsize=\footnotesize]
/******************************************************************************
 *        Métodos Numéricos para Equações Diferenciais II -- Trabalho 1       *
 *                          Ariel Nogueira Kovaljski                          *
 ******************************************************************************/

/*====================== Parâmetros a serem ajustados ========================*/

#define Lx      100.0       /* Lx: comprimento do domínio (em m) */
#define nx      50          /* nx: número de células */
#define Delta_x (Lx/nx)     /* Delta_x: largura de cada célula (em m) */
#define u_bar   0.2         /* u_bar: velocidade de escoamento (em m/s) */
#define alpha   2.0e-4      /* alpha: coeficiente de difusão */
#define c_ini   1.0         /* concentração inicial nos volumes da malha */
#define c_inj   1.5         /* concentração de injeção nos vol. da malha */
#define t_final 300.0       /* tempo final da simulação (em segundos) */
                            /* Delta_t: passo de tempo (em segundos) */
#define Delta_t (0.1 * (1/( (2*alpha)/(Delta_x*Delta_x) + u_bar/Delta_x ) ))

/*============================================================================*/

void listParameters();
void initializeArray(double arr[], int len, double value);
void calculateQ(double old_arr[], double new_arr[]);
void printAndSaveResults(double arr[], int len);
\end{Verbatim}

\subsubsection{Arquivo \texttt{main.c}}
\begin{Verbatim}[fontsize=\footnotesize]
/******************************************************************************
 *        Métodos Numéricos para Equações Diferenciais II -- Trabalho 1       *
 *                          Ariel Nogueira Kovaljski                          *
 ******************************************************************************/

#include <stdio.h>
#include <stdlib.h>
#include "main.h"

int main(void)
{

    double Q_new[nx];   /* array de Q no tempo n+1 */
    double Q_old[nx];   /* array de Q no tempo n */

    puts("\nMNED II - Trabalho 1\n====================");
    puts("por Ariel Nogueira Kovaljski\n");

    listParameters();

    /* inicializa os arrays Q para uma concentração c_ini inicial */
    initializeArray(Q_old, nx, c_ini);
    initializeArray(Q_new, nx, c_ini);

    calculateQ(Q_old, Q_new);
    printAndSaveResults(Q_new, nx);

    return 0;
}

void listParameters()
{
    puts("Parametros\n----------");
    puts("Constantes da equacao:");
    printf("Delta_t = %f, Delta_x = %f, u_bar = %3.2e, alpha = %3.2e, "
           "c_inj = %3.2e\n\n", Delta_t, Delta_t, u_bar, alpha, c_inj);
    puts("Constantes da simulacao:");
    printf("nx = %d, t_final = %f, c_ini = %3.2e\n\n", nx, t_final, c_ini);
}

/* Inicializa um array para um valor de entrada */
void initializeArray(double arr[], int len, double value)
{
    int i;
    for (i = 0; i < len; ++i) {
        arr[i] = value;
    }
}

/* Calcula as concentrações na malha ao longo do tempo */
void calculateQ(double old[], double new[])
{
    int x;
    int progress = 0, progress_count = 0;
    int progress_incr = (t_final/Delta_t) * 5 / 100;
    double t = 0;

    do {

        /*************************************************************
        *
        *            |==@==|==@==|==@==|==@==|==@==|==@==|
        *               ^
        *            Para o volume da fronteira esquerda
        *        o índice 0 refere-se ao volume nº 1 da malha
        */
        new[0] = old[0] - Delta_t/Delta_x * (
                    u_bar * (2*old[0] - 2*c_inj)
                  - alpha * (old[1] - 3*old[0] + 2*c_inj) / Delta_x
                 );

        /*************************************************************
        *
        *            |==@==|==@==|==@==|==@==|==@==|==@==|
        *                     ^     ^     ^     ^
        *             Para os volumes do centro da malha
        */
        for (x = 1; x < nx - 1; ++x) {
            new[x] = old[x] - Delta_t/Delta_x * (
                        u_bar * (old[x] - old[x-1])
                      - alpha * (old[x+1] - 2*old[x] + old[x-1]) / Delta_x
                     );
        }

        /*************************************************************
        *
        *            |==@==|==@==|==@==|==@==|==@==|==@==|
        *                                             ^
        *             Para o volume da fronteira direita
        *            x possui valor de nx - 1 nesse ponto
        */
        new[x] = old[x] - Delta_t/Delta_x * (
                    u_bar * (old[x] - old[x-1])
                  - alpha * (old[x-1] - old[x]) / Delta_x
                 );

        /* incrementa o progresso a cada 5% */
        if (progress_count == progress_incr){
            progress_count = 0;
            ++progress;
            printf("\rCalculando... %d%% concluido", progress * 5);
            fflush(stdout);
        }

        /* Atualiza array de valores antigos com os novos para a próxima
        iteração */
        for (x = 0; x < nx; ++x) {
            old[x] = new[x];
        }

        /* incrementa contador para cada 5% */
        ++progress_count;

    } while ( (t += Delta_t) <= t_final);

}

/* Imprime na tela e salva os resultados num arquivo de saída */
void printAndSaveResults(double arr[], int len)
{
    int i;
    FILE *results_file;     /* Ponteiro para o arquivo de resultados */

    /* Imprime os resultados no console */
    printf("\n\nQ[%d] (tempo final: %.2fs) = [", nx, t_final);
    for (i = 0; i < len - 1; ++i) {
        printf("%f, ", arr[i]);
    }
    printf("%f]\n\n", arr[i]);

    /* Error Handling -- Verifica se é possível criar/escrever o arquivo de
    resultados */
    if (   (results_file = fopen("./results/results.txt", "w")   ) == NULL
        && (results_file = fopen("./../results/results.txt", "w")) == NULL) {
        fputs("[ERR] Houve um erro ao escrever o arquivo \"results.txt\"! "
              "Os resultados nao foram salvos.\n", stderr);
        exit(1);
    }

    /* Adiciona os resultados no arquivo "results.txt" */
    fprintf(results_file,
            "nx=%d\n"
            "Delta_t=%f\n"
            "Delta_x=%f\n"
            "t_final=%f\n"
            "u_bar=%f\n"
            "alpha=%f\n"
            "c_ini=%f\n"
            "c_inj=%f\n",
            nx, Delta_t, Delta_x, t_final, u_bar, alpha, c_ini, c_inj);
    fputs("********************\n", results_file);

    for (i = 0; i < len; ++i) {
        fprintf(results_file, "%d,%f\n", i + 1, arr[i]);
    }

    fclose(results_file);   /* fecha o arquivo */

    puts("[INFO] Os resultados foram salvos no arquivo \"results.txt\" "
    "no diretorio \"results/\".");
}
\end{Verbatim}

\subsection{Código do Gráfico (\texttt{plot\_graph.py})}

\begin{Verbatim}[fontsize=\footnotesize]
###############################################################################
#         Métodos Numéricos para Equações Diferenciais II -- Trabalho 1       #
#                           Ariel Nogueira Kovaljski                          #
###############################################################################

import matplotlib.pyplot as plt

x = []
y = []
parameters = {}

# Abre arquivo para leitura
f = open('./results/results.txt', 'r')

for line_number, line in enumerate(f):
    if line_number < 8:
        # Adiciona parâmetros da simulação em um dicionário
        parameters[line.split('=')[0]] = (line.split('=')[1]).split('\n')[0]
    elif line_number == 8:
        # Pula linha separadora
        pass
    else:
        # Separa valores na lista `x` e na lista `y`
        x.append( int(line.split(',')[0]) )
        y.append( float( (line.split(',')[1]).split('\n')[0] ) )

# Configura e exibe o gráfico
fig,ax = plt.subplots()
fig.set_size_inches(8, 7)   # Size of the window (1in = 100px)
ax.grid(True)

plt.suptitle("Concentração X Célula")
plt.title(rf"$nx = {parameters['nx']}$, "
          rf"$\Delta t = {parameters['Delta_t']}$, "
          rf"$\Delta x = {parameters['Delta_x']}$, "
          rf"$t_{{final}} = {parameters['t_final']}$, "
          rf"$\bar{{u}} = {parameters['u_bar']}$, "
          rf"$\alpha = {parameters['alpha']}$, "
          rf"$c_{{ini}} = {parameters['c_ini']}$, "
          rf"$c_{{inj}} = {parameters['c_inj']}$", fontsize=8)
# Rótulo X -- descomentar depois
# plt.xlabel("célula (i)")

# Rótulo X -- resolvendo problema da célula
plt.xlabel("índice célula (i)")
plt.ylabel("concentração (Q)")
plt.plot(x,y,'ko-', markerfacecolor='cyan', markeredgecolor='k')
plt.show()
\end{Verbatim}
