\chapter{Introdução}
Neste trabalho foi implementado um método computacional de maneira a resolver
a equação de Advecção-Difusão de forma numérica.

Para melhor entender o desenvolvimento, é necessária introdução de alguns
conceitos-chave utilizados.

\section{A Equação de Advecção-Difusão}
A equação de advecção-difusão possibilita a solução de problemas envolvendo
variações espaciais e temporais da concentração de uma substância escoando em um
fluído. Um exemplo bastante didático consiste no despejo de esgoto em um
afluente: o contaminante sofrerá efeitos difusivos --- concentrando-se ao redor
da saída --- e efeitos advectivos --- sendo carregado no sentido da correnteza.

Para um problema unidimensional tem-se a seguinte forma,

\begin{equation} \label{adv-dif}
    \frac{\partial c}{\partial t} + \frac{\partial}{\partial x}(uc)
    - \frac{\partial}{\partial x}\left( D \frac{\partial c}{\partial x} \right)
    = 0
\end{equation}

\noindent onde $c$ indica a concentração, $u$ a velocidade e $D$ o coeficiente
de difusão.

Considerando que para $u$ e $D$ constantes, tem-se $\bar{u}$ e $\alpha$,
respectivamente, é possível reescrever \ref{adv-dif} como,

\begin{equation}
    \frac{\partial c}{\partial t} + \bar{u}\frac{\partial c}{\partial x} -
    \alpha\frac{\partial^2 c}{\partial x^2} = 0
\end{equation}

\section{Método dos Volumes Finitos}
A partir da Eq. \ref{adv-dif}, é possível reescrevê-la como

\begin{equation}
    \frac{\partial\phi}{\partial t} + \frac{\partial f}{\partial x} = 0
\end{equation}

\noindent onde,

\noindent
\begin{minipage}{.4\linewidth}
    \begin{equation}
        \phi = c
    \end{equation}
\end{minipage}%
\begin{minipage}{.6\linewidth}
    \begin{equation}
        f = f(c) = uc - D\frac{\partial c}{\partial x}
    \end{equation}
\end{minipage}


